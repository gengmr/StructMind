\documentclass[
    11pt,
    notheorems,
    hyperref={pdfauthor=whatever}
]{beamer}

% 导入演示文稿的设置
% Copyright (c) 2022 by Lars Spreng
% This work is licensed under the Creative Commons Attribution 4.0 International License.
% To view a copy of this license, visit http://creativecommons.org/licenses/by/4.0/ or send a letter to Creative Commons, PO Box 1866, Mountain View, CA 94042, USA.

%~~~~~~~~~~~~~~~~~~~~~~~~~~~~~~~~~~~~~~~~~~~~~~~~~~~~~~~~~~~~~~~~~~~~~~~~~~~~~~
% Add your packages and commands to this file
%~~~~~~~~~~~~~~~~~~~~~~~~~~~~~~~~~~~~~~~~~~~~~~~~~~~~~~~~~~~~~~~~~~~~~~~~~~~~~~

%~~~~~~~~~~~~~~~~~~~~~~~~~~~~~~~~~~~~~~~~~~~~~~~~~~~~~~~~~~~~~~~~~~~~~~~~~~~~~~
% Fonts
% \RequirePackage{palatino} % for serif slides
% \usefonttheme{serif}
\RequirePackage[scaled]{helvet} % for sans-serif slides

\RequirePackage[utf8]{inputenc}
\RequirePackage[T1]{fontenc}


\usepackage{styles/elegantmacros}
\usefolder{styles}
\usetheme[style=lecture]{elegant}

\newcommand{\makepart}[1]{%
  \part{#1}
  \frame{
    \centering
    \usebeamerfont{part title}\insertpart
  }
}

% 生成标题页命令
\newcommand{\maketitleframe}{
    \begingroup
    \makeatletter
    % 临时重定义frametitle模板,移除章节编号的插入
    \setbeamertemplate{frametitle}{
        \vspace{1cm}
        \usebeamercolor*[fg]{frametitle}\Large\insertframetitle\par\vskip1pt
        {\usebeamercolor*[fg]{framesubtitle}\small\insertframesubtitle}
        \vspace*{0.5cm}
    }

    \setbeamertemplate{footline}{}
    \begin{frame}[plain]
        \frametitle{} % 空标题
        \titlepage
    \end{frame}

    \makeatother
    \endgroup

    \addtocounter{framenumber}{-1}
}
% 生成目录页命令
\newcommand{\maketocframe}{
    \begin{frame}[shrink,label=toc]
        \frametitle{目录}
        \tableofcontents
    \end{frame}
}
% 生成致谢页命令
\newcommand{\thankspage}{
    \begingroup
    \makeatletter
    % 临时重定义frametitle模板,移除章节编号的插入
    \setbeamertemplate{frametitle}{
        \vspace{1cm}
        \usebeamercolor*[fg]{frametitle}\Large\insertframetitle\par\vskip1pt
        {\usebeamercolor*[fg]{framesubtitle}\small\insertframesubtitle}
        \vspace*{0.5cm}
    }
    % 创建帧
    \begin{frame}
        \frametitle{} % 空标题
        \begin{center}
            {\Huge\calligra Thank You}
        \end{center}
    \end{frame}
    \makeatother
    \endgroup
}
%~~~~~~~~~~~~~~~~~~~~~~~~~~~~~~~~~~~~~~~~~~~~~~~~~~~~~~~~~~~~~~~~~~~~~~~~~~~~~~

%~~~~~~~~~~~~~~~~~~~~~~~~~~~~~~~~~~~~~~~~~~~~~~~~~~~~~~~~~~~~~~~~~~~~~~~~~~~~~~
% Figures
\RequirePackage{booktabs}
\RequirePackage{colortbl}
\RequirePackage{ragged2e}
\RequirePackage{schemabloc}
%\RequirePackage{natbib}
\RequirePackage{caption}
\RequirePackage{subcaption}
\RequirePackage{tabularx}
\RequirePackage{array}
\RequirePackage{multirow}
\RequirePackage[%
  natbib=true, backend=biber,%
  style=apa, isbn=false,url=false,uniquename=false%, useprefix=true%
  ]{biblatex}
\addbibresource{references.bib}
\newcolumntype{Y}{>{\centering\arraybackslash}X}

%~~~~~~~~~~~~~~~~~~~~~~~~~~~~~~~~~~~~~~~~~~~~~~~~~~~~~~~~~~~~~~~~~~~~~~~~~~~~~~

%~~~~~~~~~~~~~~~~~~~~~~~~~~~~~~~~~~~~~~~~~~~~~~~~~~~~~~~~~~~~~~~~~~~~~~~~~~~~~~
% Figures
\RequirePackage{wrapfig}
\RequirePackage{pgfplots}
\RequirePackage{graphicx}
\RequirePackage{adjustbox}
\RequirePackage{environ}
\pgfplotsset{compat=1.18}

\makeatletter
\newsavebox{\measure@tikzpicture}
\NewEnviron{scaletikzpicturetowidth}[1]{%
  \def\tikz@width{#1}%
  \def\tikzscale{1}\begin{lrbox}{\measure@tikzpicture}%
  \BODY
  \end{lrbox}%
  \pgfmathparse{#1/\wd\measure@tikzpicture}%
  \edef\tikzscale{\pgfmathresult}%
  \BODY
}
\makeatother
%~~~~~~~~~~~~~~~~~~~~~~~~~~~~~~~~~~~~~~~~~~~~~~~~~~~~~~~~~~~~~~~~~~~~~~~~~~~~~~

%~~~~~~~~~~~~~~~~~~~~~~~~~~~~~~~~~~~~~~~~~~~~~~~~~~~~~~~~~~~~~~~~~~~~~~~~~~~~~~
% Maths
\RequirePackage{textcomp}
\RequirePackage{amsmath}
\RequirePackage{amsthm}
\RequirePackage{mathtools}
%\RequirePackage{bbm}
%\RequirePackage{algorithm}
%\RequirePackage[osf,sc]{mathpazo}
%\RequirePackage{pifont}
%\newcommand{\xmark}{\ding{55}}%
%\numberwithin{equation}{section}
\DeclareMathOperator*{\argmax}{arg\,max}
\DeclareMathOperator*{\argmin}{arg\,min}

\setbeamertemplate{theorems}[numbered] % to number

\theoremstyle{definition}
\newtheorem{fact}{Fact}[section]
\newtheorem{examp}{Example}[section]

\theoremstyle{plain}
\newtheorem{definition}{Definition}[section]
\newtheorem{proposition}{Proposition}
\newtheorem{theorem}{Theorem}
\newtheorem{assumption}{Assumption}

\providecommand{\H}{\mathscr{H}}
\providecommand{\E}{\mathbb{E}}
\makeatletter
\def\munderbar#1{\underline{\sbox\tw@{$#1$}\dp\tw@\z@\box\tw@}}
\makeatother
%~~~~~~~~~~~~~~~~~~~~~~~~~~~~~~~~~~~~~~~~~~~~~~~~~~~~~~~~~~~~~~~~~~~~~~~~~~~~~~
% 导入必要的包
\usepackage{fontspec}
\usepackage{xeCJK}
\usepackage{calligra}
\usepackage{tabularx}
\usepackage{colortbl}

% PingFang 字体定义
\setCJKmainfont{PingFang.ttf}[
    Path = fonts/,
    Extension = .ttf,
    UprightFont = *,
    BoldFont = PingFang-bold,
    ItalicFont = *,
    BoldItalicFont = *
]



% 设置文档的标题、副标题和作者信息
\title{文档标题:xx}
\subtitle{文档副标题:xx}
\author{作者:xxx}
\date{\today}

\begin{document}

% 生成标题页
\maketitleframe
% 生成目录页
\maketocframe

% 介绍部分
\section{Introduction}

\begin{frame}
    \begin{itemize}
        \item 这个模板提供了优雅且简洁的布局。
        \item 本模板为不满意现有模板的用户设计。
        \item 目标是提供一个简单但美观的布局,注重内容展示。
        \item 适用于讲座笔记和技术演示,也适用于任何类型的演讲。
    \end{itemize}
\end{frame}

% 颜色部分
\subsection{Colors}
\begin{frame}[fragile]
    \centering
    模板提供不同的颜色主题。\\
    在loadslides.tex中设置 \verb+\usetheme[style=lecture]{elegant}+\\[0.8cm]
    \newcommand{\colorRow}[1]{
        \begin{tabular}{p{4cm}cccc}
        #1 & \cellcolor{primary}\hspace*{1cm} &\cellcolor{secondary}\hspace*{1cm}&\cellcolor{tertiary}\hspace*{1cm}\\
        \end{tabular}
    }
    % 示例颜色主题
    \colorRow{Lecture}\\[0.3cm]
    \definecolor{primary}{HTML}{08457E}
    \definecolor{secondary}{HTML}{B8860B}
    \definecolor{tertiary}{HTML}{B22222}
    \colorRow{Gold}\\[0.3cm]
    \definecolor{primary}{HTML}{08457E}
    \definecolor{secondary}{HTML}{B02A30}
    \definecolor{tertiary}{HTML}{006400}
    \colorRow{Red}\\[0.3cm]
    \definecolor{primary}{HTML}{08457E}
    \definecolor{secondary}{HTML}{CC5500}
    \definecolor{tertiary}{HTML}{008B8B}
    \colorRow{Orange}\\[0.3cm]
    \definecolor{primary}{HTML}{08457E}
    \definecolor{secondary}{HTML}{E60073}
    \definecolor{tertiary}{HTML}{33CCCC}
    \colorRow{Gray}\\[0.3cm]

    \centering Gray 是默认 \textit{Lecture} 主题的低调版本,使用灰色替代粉色字幕。
\end{frame}

% Frames 部分
\subsection{Frames}
\begin{frame}
    除非用户自定义帧标题和副标题,否则 Elegant Slides 会自动插入节标题和子节标题作为帧标题和帧副标题。
\end{frame}

\begin{frame}{}{Custom Subsection}
    该帧具有自定义副标题。帧标题自动插入,对应于节标题。
\end{frame}

\begin{frame}{Custom Title}{Custom Subsection with Footnote}
    该帧具有自定义标题和自定义副标题。\footnote{这是一个脚注。另请参见 \citet{example_2022}。}
\end{frame}

% Fonts 部分
\subsection{Fonts}
\begin{frame}[fragile]
    \begin{itemize}
        \item 字体类型可在 loadslides.tex 中更改。
        \item 对于讲座笔记或报告,\textit{Lecture} 主题结合 \verb+\RequirePackage{palatino}+ 和 \verb+\usefonttheme{serif}+ 效果很好。
        \item 对于讲座和其他演示,\verb+\RequirePackage[scaled]{helvet}+ 与其他主题(如 \textit{Gold})搭配更佳。
        \item 文本突出显示方式如下:
        \begin{itemize}
            \item 常规
            \item \emphasize{强调}:这是一个例子,\textbf{强调}某事,\alert{警示}不要做某事
            \item \alert{警示}
            \item \example{示例}
            \item \textit{斜体}
            \item \textbf{粗体}
        \end{itemize}
    \end{itemize}
\end{frame}

% Lists 部分
\subsection{Lists}
\begin{frame}
    \begin{columns}[T,onlytextwidth]
        \column{0.33\textwidth}
            \textbf{项目}
            \begin{itemize}
                \item 猫
                \begin{itemize}
                    \item 英国短毛猫
                \end{itemize}
                \item 狗
                \item 鸟
            \end{itemize}

        \column{0.33\textwidth}
            \textbf{枚举}
            \begin{enumerate}
                \item 第一点
                \begin{enumerate}
                    \item 第一子点
                \end{enumerate}
                \item 第二点
                \item 最后一点
            \end{enumerate}

        \column{0.33\textwidth}
            \textbf{描述}
            \begin{description}
                \item[苹果] 是
                \item[橘子] 否
                \item[葡萄] 否
            \end{description}
    \end{columns}
\end{frame}

% Table 部分
\subsection{Table}
\begin{frame}
    \begin{table}
        \caption{世界上最大的城市(来源:维基百科)}
        \begin{tabular}{@{} lr @{}}
            \toprule
            城市 & 人口 \\
            \midrule
            墨西哥城 & 20,116,842 \\
            上海 & 19,210,000 \\
            北京 & 15,796,450 \\
            伊斯坦布尔 & 14,160,467 \\
            \bottomrule
        \end{tabular}
        \hspace*{1cm}
            \setlength\extrarowheight{3pt}
        \begin{tabular}{|lr|}
            \hline
            \rowcolor{primary}\color{white}城市 & \color{white}人口 \\
            \hline
            墨西哥城 & 20,116,842 \\
            上海 & 19,210,000 \\
            北京 & 15,796,450 \\
            伊斯坦布尔 & 14,160,467 \\
            \hline
        \end{tabular}
    \end{table}
\end{frame}


% Figures 部分
\subsection{Figures}
\begin{frame}
    \begin{figure}[htbp]
        \centering
        \caption{Plot of $y=x^2$}
        \begin{tikzpicture}
            \begin{axis}[
                legend columns=3,
                legend style={at={(0.5,-0.3)},anchor=north},
                width = \textwidth,
                height = 2.5in,
                xmin = -3,
                xmax = 3,
                ymin = 0,
                ymax = 10,
            ]
                \addplot[primary] {x^2};
                \addlegendentry{$x^2$}
            \end{axis}
        \end{tikzpicture}
    \end{figure}
\end{frame}

% Blocks 部分
\subsection{Blocks}
\begin{frame}
    \centering
    \begin{minipage}[b]{0.5\textwidth}
        \begin{block}{默认}
            区块内容。
        \end{block}

        \begin{alertblock}{警示}
            区块内容。
        \end{alertblock}

        \begin{exampleblock}{示例}
            区块内容。
        \end{exampleblock}
    \end{minipage}
\end{frame}

% Maths 部分
\section{Maths}
\subsection{Equations}
\begin{frame}
    \begin{itemize}
        \item 一个编号的方程:
        \begin{equation}
            y_t = \beta x_t + \varepsilon_t
        \end{equation}
        \item 另一个方程:
        \begin{equation*}
            \mathbf{Y} = \boldsymbol{\beta} \mathbf{X} + \boldsymbol{\varepsilon}_t
        \end{equation*}
    \end{itemize}
\end{frame}

% Theorem 部分
\subsection{Theorem}
\begin{frame}
    \begin{itemize}
        \item 定理按连续编号。
    \end{itemize}
    \begin{theorem}[示例定理]
        对于一个离散随机变量 X,它在字母表 \(\mathcal{X}\) 中取值,并根据 \(p:{\mathcal {X}}\to [0,1]\) 分布:
        \begin{equation}
            \mathrm {H} (X):=-\sum _{x\in {\mathcal {X}}}p(x)\log p(x)=\mathbb {E} [-\log p(X)]
        \end{equation}
    \end{theorem}
\end{frame}

% Definition 部分
\begin{frame}{}{Definitions}
    \begin{itemize}
        \item 定义编号前缀为相应部分的节号。
    \end{itemize}
    \begin{definition}[示例定义]
        对于一个离散随机变量 X,它在字母表 \(\mathcal{X}\) 中取值,并根据 \(p:{\mathcal {X}}\to [0,1]\) 分布:
        \begin{equation}
            \mathrm {H} (X):=-\sum _{x\in {\mathcal {X}}}p(x)\log p(x)=\mathbb {E} [-\log p(X)]
        \end{equation}
    \end{definition}
\end{frame}

% References 部分
\begin{frame}[allowframebreaks]{References}
    \printbibliography
\end{frame}

% 致谢页
\thankspage

\end{document}
