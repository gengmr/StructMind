% Copyright (c) 2022 by Lars Spreng
% This work is licensed under the Creative Commons Attribution 4.0 International License.
% To view a copy of this license, visit http://creativecommons.org/licenses/by/4.0/ or send a letter to Creative Commons, PO Box 1866, Mountain View, CA 94042, USA.

%~~~~~~~~~~~~~~~~~~~~~~~~~~~~~~~~~~~~~~~~~~~~~~~~~~~~~~~~~~~~~~~~~~~~~~~~~~~~~~
% Add your packages and commands to this file
%~~~~~~~~~~~~~~~~~~~~~~~~~~~~~~~~~~~~~~~~~~~~~~~~~~~~~~~~~~~~~~~~~~~~~~~~~~~~~~

%~~~~~~~~~~~~~~~~~~~~~~~~~~~~~~~~~~~~~~~~~~~~~~~~~~~~~~~~~~~~~~~~~~~~~~~~~~~~~~
% Fonts
% \RequirePackage{palatino} % for serif slides
% \usefonttheme{serif}
\RequirePackage[scaled]{helvet} % for sans-serif slides

\RequirePackage[utf8]{inputenc}
\RequirePackage[T1]{fontenc}


\usepackage{styles/elegantmacros}
\usefolder{styles}
\usetheme[style=lecture]{elegant}

\newcommand{\makepart}[1]{%
  \part{#1}
  \frame{
    \centering
    \usebeamerfont{part title}\insertpart
  }
}

% 生成标题页命令
\newcommand{\maketitleframe}{
    \begingroup
    \makeatletter
    % 临时重定义frametitle模板,移除章节编号的插入
    \setbeamertemplate{frametitle}{
        \vspace{1cm}
        \usebeamercolor*[fg]{frametitle}\Large\insertframetitle\par\vskip1pt
        {\usebeamercolor*[fg]{framesubtitle}\small\insertframesubtitle}
        \vspace*{0.5cm}
    }

    \setbeamertemplate{footline}{}
    \begin{frame}[plain]
        \frametitle{} % 空标题
        \titlepage
    \end{frame}

    \makeatother
    \endgroup

    \addtocounter{framenumber}{-1}
}
% 生成目录页命令
\newcommand{\maketocframe}{
    \begin{frame}[shrink,label=toc]
        \frametitle{目录}
        \tableofcontents
    \end{frame}
}
% 生成致谢页命令
\newcommand{\thankspage}{
    \begingroup
    \makeatletter
    % 临时重定义frametitle模板,移除章节编号的插入
    \setbeamertemplate{frametitle}{
        \vspace{1cm}
        \usebeamercolor*[fg]{frametitle}\Large\insertframetitle\par\vskip1pt
        {\usebeamercolor*[fg]{framesubtitle}\small\insertframesubtitle}
        \vspace*{0.5cm}
    }
    % 创建帧
    \begin{frame}
        \frametitle{} % 空标题
        \begin{center}
            {\Huge\calligra Thank You}
        \end{center}
    \end{frame}
    \makeatother
    \endgroup
}
%~~~~~~~~~~~~~~~~~~~~~~~~~~~~~~~~~~~~~~~~~~~~~~~~~~~~~~~~~~~~~~~~~~~~~~~~~~~~~~

%~~~~~~~~~~~~~~~~~~~~~~~~~~~~~~~~~~~~~~~~~~~~~~~~~~~~~~~~~~~~~~~~~~~~~~~~~~~~~~
% Figures
\RequirePackage{booktabs}
\RequirePackage{colortbl}
\RequirePackage{ragged2e}
\RequirePackage{schemabloc}
%\RequirePackage{natbib}
\RequirePackage{caption}
\RequirePackage{subcaption}
\RequirePackage{tabularx}
\RequirePackage{array}
\RequirePackage{multirow}
\RequirePackage[%
  natbib=true, backend=biber,%
  style=apa, isbn=false,url=false,uniquename=false%, useprefix=true%
  ]{biblatex}
\addbibresource{references.bib}
\newcolumntype{Y}{>{\centering\arraybackslash}X}

%~~~~~~~~~~~~~~~~~~~~~~~~~~~~~~~~~~~~~~~~~~~~~~~~~~~~~~~~~~~~~~~~~~~~~~~~~~~~~~

%~~~~~~~~~~~~~~~~~~~~~~~~~~~~~~~~~~~~~~~~~~~~~~~~~~~~~~~~~~~~~~~~~~~~~~~~~~~~~~
% Figures
\RequirePackage{wrapfig}
\RequirePackage{pgfplots}
\RequirePackage{graphicx}
\RequirePackage{adjustbox}
\RequirePackage{environ}
\pgfplotsset{compat=1.18}

\makeatletter
\newsavebox{\measure@tikzpicture}
\NewEnviron{scaletikzpicturetowidth}[1]{%
  \def\tikz@width{#1}%
  \def\tikzscale{1}\begin{lrbox}{\measure@tikzpicture}%
  \BODY
  \end{lrbox}%
  \pgfmathparse{#1/\wd\measure@tikzpicture}%
  \edef\tikzscale{\pgfmathresult}%
  \BODY
}
\makeatother
%~~~~~~~~~~~~~~~~~~~~~~~~~~~~~~~~~~~~~~~~~~~~~~~~~~~~~~~~~~~~~~~~~~~~~~~~~~~~~~

%~~~~~~~~~~~~~~~~~~~~~~~~~~~~~~~~~~~~~~~~~~~~~~~~~~~~~~~~~~~~~~~~~~~~~~~~~~~~~~
% Maths
\RequirePackage{textcomp}
\RequirePackage{amsmath}
\RequirePackage{amsthm}
\RequirePackage{mathtools}
%\RequirePackage{bbm}
%\RequirePackage{algorithm}
%\RequirePackage[osf,sc]{mathpazo}
%\RequirePackage{pifont}
%\newcommand{\xmark}{\ding{55}}%
%\numberwithin{equation}{section}
\DeclareMathOperator*{\argmax}{arg\,max}
\DeclareMathOperator*{\argmin}{arg\,min}

\setbeamertemplate{theorems}[numbered] % to number

\theoremstyle{definition}
\newtheorem{fact}{Fact}[section]
\newtheorem{examp}{Example}[section]

\theoremstyle{plain}
\newtheorem{definition}{Definition}[section]
\newtheorem{proposition}{Proposition}
\newtheorem{theorem}{Theorem}
\newtheorem{assumption}{Assumption}

\providecommand{\H}{\mathscr{H}}
\providecommand{\E}{\mathbb{E}}
\makeatletter
\def\munderbar#1{\underline{\sbox\tw@{$#1$}\dp\tw@\z@\box\tw@}}
\makeatother
%~~~~~~~~~~~~~~~~~~~~~~~~~~~~~~~~~~~~~~~~~~~~~~~~~~~~~~~~~~~~~~~~~~~~~~~~~~~~~~
% 导入必要的包
\usepackage{fontspec}
\usepackage{xeCJK}
\usepackage{calligra}
\usepackage{tabularx}
\usepackage{colortbl}

% PingFang 字体定义
\setCJKmainfont{PingFang.ttf}[
    Path = fonts/,
    Extension = .ttf,
    UprightFont = *,
    BoldFont = PingFang-bold,
    ItalicFont = *,
    BoldItalicFont = *
]

